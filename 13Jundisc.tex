\documentclass[oneside,11pt]{amsart}
\usepackage{capt-of}
\usepackage{xcolor}
\usepackage{fullpage}
\usepackage{amsmath}
\usepackage{amsfonts}
\usepackage{amssymb}
\usepackage{amsthm}
\usepackage{graphicx}
\usepackage{tikz-cd}
\usepackage{float} % To forcefully place figures at a particular place using the H option
\usepackage{quiver}
\usepackage{caption} % For empty captions without the colon in figures
\usepackage{enumitem}
\usepackage{nicefrac}
\usepackage{hyperref}
\usepackage[capitalise]{cleveref}
\usepackage[mathscr]{euscript}
\usepackage{mlmodern}
\usepackage[T1]{fontenc}
\usepackage[]{mdframed}

%\newtheorem{thm}{Theorem}[section]
\newtheorem{thm}{Theorem}
\crefname{thm}{Theorem}{Theorems}
\newtheorem{prop}[thm]{Proposition}
\crefname{prop}{Proposition}{Propositions}
\newtheorem{lem}[thm]{Lemma}
\crefname{lem}{Lemma}{Lemmas}
\newtheorem{cor}[thm]{Corollary}
\crefname{cor}{Corollary}{Corollaries}
\theoremstyle{definition}
\newtheorem{ex}[thm]{Example}
\crefname{ex}{Example}{Examples}
\newtheorem{defn}[thm]{Definition}
\crefname{defn}{Definition}{Definitions}

\newtheorem{const}[thm]{Construction}
\newtheorem{conj}[thm]{Conjecture}
\crefname{conj}{Conjecture}{Conjectures}

\newtheorem{obs}[thm]{Observation}
\crefname{obs}{Observation}{Observations}

\newtheorem{notn}[thm]{Notation}
\crefname{notn}{Notation}{Notation}
\newtheorem{conv}[thm]{Convention}
\crefname{conv}{Convention}{Conventions}
\theoremstyle{remark}
\newtheorem{rem}[thm]{Remark}
\crefname{rem}{Remark}{Remarks}



\begin{document}
\section*{Discussion on 13 June 2025} 
\begin{quotation}
 The key goal here is to eventually prove the following results:
 \begin{itemize}
	 \item Direct sums of exact sequences of (pre)sheaves on a topological space are exact. (\cref{prop1,prop2})
	 \item Direct products of epimorphisms of sheaves on $X$ are not necessarily epimorphisms. (\cref{const1})
	 \item Let $\mathscr{T} $ be a triangulated category and $\mathscr{L} $ be a localizing subcategory of $\mathscr{T} $. If $f:x\rightarrow y$ be a morphism in $\mathscr{T} $ that is mapped to $0$ by the projection $\mathscr{T} \rightarrow \mathscr{T}_{/\mathscr{L}}   $, then $f$ factors through an object of $\mathscr{L} $. (\cref{proptri})
 \end{itemize}
\end{quotation}
\begin{notn}
For a topological space $X$, $\textsf{Sh}(X)$ (resp. $\textsf{PSh}(X)$) denotes the category of sheaves (resp. presheaves) of abelian groups over $X$.      
\end{notn}

\begin{prop}\label{prop1}
 Let $X$ be a topological space and $I$ be a set. If for each $i\in I$,
 \begin{equation*}
 0 \longrightarrow  \mathscr{F}^{i} \longrightarrow \mathscr{G}^{i} \longrightarrow \mathscr{H}^{i} \longrightarrow 0  
 \end{equation*}
 is an exact sequence in $\emph{\textsf{PSh}}(X)$, then,
 \begin{equation*}
   0 \longrightarrow  \bigoplus_{i}\mathscr{F}^{i} \longrightarrow \bigoplus_{i}\mathscr{G}^{i} \longrightarrow \bigoplus_{i}\mathscr{H}^{i} \longrightarrow 0 
 \end{equation*}
 is also exact.
\end{prop}
\begin{proof}
By assumption, for any open subset $U$ of $X$,
\begin{equation*}
 0 \longrightarrow \mathscr{F}^{i}(U)  \longrightarrow \mathscr{G}^{i}(U) \longrightarrow \mathscr{H}^{i}(U) \longrightarrow 0
\end{equation*}
is exact. Since direct sums of exact sequences are exact in the category $\textsf{Ab}$ of abelian groups, the sequence
\begin{equation*}
 0 \longrightarrow \left(\bigoplus_{i} \mathscr{F}^{i}\right)(U)  \longrightarrow  \left(\bigoplus_{i}\mathscr{G}^{i}\right)(U) \longrightarrow \left(\bigoplus_{i}\mathscr{H}^{i}\right)(U) \longrightarrow 0
\end{equation*}
is exact. Thus, it follows that
\begin{equation*}
 0 \longrightarrow \bigoplus_{i}\mathscr{F}^{i} \longrightarrow \bigoplus_{i}\mathscr{G}^{i} \longrightarrow \bigoplus_{i}\mathscr{H}^{i} \longrightarrow 0
\end{equation*}
is exact.
\end{proof}
\begin{prop}\label{prop2}
 Let $X$ be a topological space and $I$ be a set. If for each $i\in I$,
 \begin{equation*}
 0 \longrightarrow  \mathscr{F}^{i} \longrightarrow \mathscr{G}^{i} \longrightarrow \mathscr{H}^{i} \longrightarrow 0  
 \end{equation*}
 is an exact sequence in $\emph{\textsf{Sh}}(X)$, then,
 \begin{equation}\label{eq2}
   0 \longrightarrow  \bigoplus_{i}\mathscr{F}^{i} \longrightarrow \bigoplus_{i}\mathscr{G}^{i} \longrightarrow \bigoplus_{i}\mathscr{H}^{i} \longrightarrow 0 
 \end{equation}
 is also exact.
\end{prop}
\begin{proof}
	Let $U:\textsf{Sh}(X) \rightarrow \textsf{PSh}(X) $ denote the forgetful functor and ${\_}^{\dag}: \textsf{Sh}(X)  \rightarrow  \textsf{PSh}(X)$ denote the sheafification. Then, the sequence in \cref{eq2} is isomorphic to the following sequence
	\begin{equation*}
	 0 \rightarrow (\bigoplus_{i} U(\mathscr{F}^{i}))^{\dag} \rightarrow (\bigoplus_{i} U(\mathscr{G}^{i}))^{\dag} \rightarrow (\bigoplus_{i} U(\mathscr{H}^{i}))^{\dag} \rightarrow 0
	\end{equation*}
	For any $x\in X$, since the forgetful and sheafification functors preserve stalks and since direct sums commute with colimits, applying the stalk functor $(\_)_{x}$ yields (upto isomorphism) the following sequence
\begin{equation*}
 0 \longrightarrow  \bigoplus_{i}\mathscr{F}^{i}_{x} \longrightarrow \bigoplus_{i}\mathscr{G}^{i}_{x} \longrightarrow \bigoplus_{i}\mathscr{H}^{i}_{x} \longrightarrow 0 
\end{equation*}
which is exact since direct sums of exact sequences are exact in $\textsf{Ab} $. 
\end{proof}

\begin{const}\label{const1}
	Define $$X:=\{ 0 \} \cup \left\{ \frac{1}{n} \;|\;n\in \mathbb{N}  \right\} \subseteq \mathbb{R} $$ 
and
$$ \tau:=\{ [0,m)\cap X\;|\;m\in \mathbb{R}_{\geqslant 0} \} \cup \{ (0,m)\cap X\;|\;m\in \mathbb{R}_{\geqslant 0}  \}   $$
It is easy to see that $(X,\tau)$ is a topological space. For each $n\in \mathbb{N} $, define a presheaf $\mathscr{F}_{n}$ on $X$ as follows
\begin{align*}
\mathscr{F}_{n}(U):=\begin{cases}
	\mathbb{Z}  &\text{ if } \quad \emptyset \neq U\subseteq [0,1/n) \\
	0  & \text{ otherwise} 
\end{cases}
\end{align*}
and the restriction map $\mathscr{F}_{n}(U)\rightarrow \mathscr{F}_{n}(V) $ is defined to be $\text{Id}_{\mathbb{Z} }$ if $\emptyset \neq V \subseteq U\subseteq [0,\frac{1}{n})$ and $0$ otherwise. In fact, $\mathscr{F}_{n}$ is further a sheaf. Base identity is clear and it is sufficient to check base gluability on open covers of the form $[0,\frac{1}{m'})=[0,\frac{1}{m})\cup (0,\frac{1}{m'})$ where $m'<m$.

Let $\mathscr{G}:=\textsf{Sky}_{0}(\mathbb{Z} ) $, the skyscraper sheaf at $0$ over $X$ with stalk $\mathbb{Z} $ at $0$. By adjointness, the identity map $(\mathscr{F}_{n})_{0}= \mathbb{Z}  \rightarrow \mathbb{Z} $ induces a surjective map of sheaves $\phi _{n}:\mathscr{F}_{n}\rightarrow \mathscr{G} $ for each $n\in \mathbb{N} $. We claim that $\prod_{i}\phi _{i}: \prod_{i}\mathscr{F}_{i} \rightarrow \prod_{i}\mathscr{G}   $ is not surjective.   It is easy to see that $(\prod_{i}\mathscr{G})_{0}\cong \prod_{i}\mathbb{Z} $ which is uncountable. However, for any given open neighbourhood $U$ of $0$, $\mathscr{F}_{i}(U)$ is $0$ for large enough $i$. Thus, every abelian group appearing in the countable colimit defining the stalk $(\prod_{i}\mathscr{F}_{i})_{0}$ is countable and consequently, the stalk $(\prod_{i}(\mathscr{F}_{i} ))_{0}$ is itself countable. This proves that $\prod_{i}\phi _{i}$ is not surjective.\footnote{The countability argument to simplify the proof of $\prod_{i}\phi _{i}$ not being surjective is due to a friend, Atharva Raje, who listened to an earlier and dirtier version of the proof}.     
\end{const}







\begin{prop}\label{proptri}
 Let $\mathscr{T} $ be a triangulated category, $\mathscr{L} $ a localising subcategory and $\mathscr{T}_{/\mathscr{L} }$ denote the Verdier quotient. A morphism $f:x \rightarrow y $ in $\mathscr{T} $ is mapped to the zero morphism by the projection $\mathscr{T} \rightarrow \mathscr{T}_{/\mathscr{L} }  $ if and only if $f$ factors through an object in $\mathscr{L} $.
\end{prop}
\begin{proof}
 Suppose that $f$ is taken to $0$ by the projection. That is, there exist $z,z'\in \mathscr{T} $, quasi-isomorphisms (relative to $\mathscr{L} $) $\sigma ,\sigma '$ and morphisms $f',g,h$ in $\mathscr{T} $ making the following diagram commute.      
 % https://q.uiver.app/#q=WzAsNSxbMCwyLCJ4Il0sWzIsMiwieiciXSxbNCwyLCJ5Il0sWzIsMCwieSJdLFsyLDQsInoiXSxbMCwxLCJmJyJdLFsyLDEsIlxcc2lnbWEnIiwyXSxbMiwzLCIiLDAseyJsZXZlbCI6Miwic3R5bGUiOnsiaGVhZCI6eyJuYW1lIjoibm9uZSJ9fX1dLFswLDMsImYiXSxbMywxLCJnIiwxXSxbMCw0LCIwIiwyXSxbNCwxLCJoIiwxXSxbMiw0LCJcXHNpZ21hIl0sWzIsNCwiXFxzaW0iLDIseyJzdHlsZSI6eyJib2R5Ijp7Im5hbWUiOiJub25lIn0sImhlYWQiOnsibmFtZSI6Im5vbmUifX19XSxbMiwxLCJcXHNpbSIsMCx7InN0eWxlIjp7ImJvZHkiOnsibmFtZSI6Im5vbmUifSwiaGVhZCI6eyJuYW1lIjoibm9uZSJ9fX1dXQ==
\[\begin{tikzcd}
	&& y \\
	\\
	x && {z'} && y \\
	\\
	&& z
	\arrow["g"{description}, from=1-3, to=3-3]
	\arrow["f", from=3-1, to=1-3]
	\arrow["{f'}", from=3-1, to=3-3]
	\arrow["0"', from=3-1, to=5-3]
	\arrow[equals, from=3-5, to=1-3]
	\arrow["{\sigma'}"', from=3-5, to=3-3]
	\arrow["\sim", draw=none, from=3-5, to=3-3]
	\arrow["\sigma", from=3-5, to=5-3]
	\arrow["\sim"', draw=none, from=3-5, to=5-3]
	\arrow["h"{description}, from=5-3, to=3-3]
\end{tikzcd}\]
Equivalently, there exists a quasi-isomorphism $g$ such that $g\circ f=0$. Hence, there exists a distinguished triangle of the following form in $\mathscr{T} $
% https://q.uiver.app/#q=WzAsNCxbMCwwLCJ5Il0sWzIsMCwieiJdLFs0LDAsInciXSxbNiwwLCJcXFNpZ21hIHkiXSxbMCwxLCJnIl0sWzEsMiwiXFxhbHBoYSJdLFsyLDMsIlxcYmV0YSJdXQ==
\[\begin{tikzcd}
	y && z && w && {\Sigma y}
	\arrow["g", from=1-1, to=1-3]
	\arrow["\alpha", from=1-3, to=1-5]
	\arrow["\beta", from=1-5, to=1-7]
\end{tikzcd}\]
where $w\in \mathscr{L} $. By one of the axioms defining a (pre)triangulated category, there exists a dotted arrow $x \rightarrow \Sigma^{-1}w$ making the following diagram commute, thus yielding the required factorisation.
% https://q.uiver.app/#q=WzAsOCxbMCwyLCJcXFNpZ21hXnstMX15Il0sWzIsMiwiXFxTaWdtYV57LTF9eiJdLFs0LDIsIlxcU2lnbWFeey0xfXciXSxbNiwyLCJ5Il0sWzAsMCwiXFxTaWdtYV57LTF9eCJdLFsyLDAsIjAiXSxbNCwwLCJ4Il0sWzYsMCwieCJdLFswLDEsIlxcU2lnbWFeey0xfWciXSxbMiwzLCItXFxTaWdtYV57LTF9XFxiZXRhIl0sWzEsMiwiXFxTaWdtYV57LTF9XFxhbHBoYSJdLFs0LDAsIlxcU2lnbWFeey0xfWYiLDJdLFs0LDVdLFs1LDZdLFs2LDcsIlxcdGV4dHtJZH0iXSxbNywzLCJmIl0sWzYsMiwiIiwxLHsic3R5bGUiOnsiYm9keSI6eyJuYW1lIjoiZGFzaGVkIn19fV0sWzUsMSwiMCIsMV1d
\[\begin{tikzcd}
	{\Sigma^{-1}x} && 0 && x && x \\
	\\
	{\Sigma^{-1}y} && {\Sigma^{-1}z} && {\Sigma^{-1}w} && y
	\arrow[from=1-1, to=1-3]
	\arrow["{\Sigma^{-1}f}"', from=1-1, to=3-1]
	\arrow[from=1-3, to=1-5]
	\arrow["0"{description}, from=1-3, to=3-3]
	\arrow["{\text{Id}}", from=1-5, to=1-7]
	\arrow[dashed, from=1-5, to=3-5]
	\arrow["f", from=1-7, to=3-7]
	\arrow["{\Sigma^{-1}g}", from=3-1, to=3-3]
	\arrow["{\Sigma^{-1}\alpha}", from=3-3, to=3-5]
	\arrow["{-\Sigma^{-1}\beta}", from=3-5, to=3-7]
\end{tikzcd}\]
Conversely, suppose that $f$ factors as in the following diagram, where $w\in L$.
% https://q.uiver.app/#q=WzAsMyxbMCwwLCJ4Il0sWzIsMCwieSJdLFsxLDEsInciXSxbMCwxLCJmIl0sWzAsMiwiXFxhbHBoYSIsMl0sWzIsMSwiXFxiZXRhIiwyXV0=
\[\begin{tikzcd}
	x && y \\
	& w
	\arrow["f", from=1-1, to=1-3]
	\arrow["\alpha"', from=1-1, to=2-2]
	\arrow["\beta"', from=2-2, to=1-3]
\end{tikzcd}\]

As previously argued, it is sufficient to show that there exists a quasi-isomorphism $g$ such that $g\circ f=0$. By one of the axioms defining a (pre)triangulated category, there exists a diagram of the following form, where the rows are distinguished and there is a dotted arrow that completes the morphism of distinguished triangles.
% https://q.uiver.app/#q=WzAsOCxbMCwwLCJ4Il0sWzIsMCwieSJdLFswLDIsInciXSxbMiwyLCJ5Il0sWzQsMCwiXFx0ZXh0e2NvbmV9KGYpIl0sWzYsMCwiXFxTaWdtYSB4Il0sWzYsMiwiXFxTaWdtYSB3Il0sWzQsMiwiXFx0ZXh0e2NvbmV9KFxcYmV0YSkiXSxbMCwxLCJmIl0sWzAsMiwiXFxhbHBoYSIsMl0sWzIsM10sWzEsMywiIiwwLHsibGV2ZWwiOjIsInN0eWxlIjp7ImhlYWQiOnsibmFtZSI6Im5vbmUifX19XSxbMSw0XSxbNCw1XSxbNSw2XSxbMyw3XSxbNyw2XSxbNCw3LCIiLDEseyJzdHlsZSI6eyJib2R5Ijp7Im5hbWUiOiJkYXNoZWQifX19XV0=
\[\begin{tikzcd}
	x && y && {\text{cone}(f)} && {\Sigma x} \\
	\\
	w && y && {\text{cone}(\beta)} && {\Sigma w}
	\arrow["f", from=1-1, to=1-3]
	\arrow["\alpha"', from=1-1, to=3-1]
	\arrow[from=1-3, to=1-5]
	\arrow[equals, from=1-3, to=3-3]
	\arrow[from=1-5, to=1-7]
	\arrow[dashed, from=1-5, to=3-5]
	\arrow["\Sigma \alpha",from=1-7, to=3-7]
	\arrow["\beta ",from=3-1, to=3-3]
	\arrow[from=3-3, to=3-5]
	\arrow[from=3-5, to=3-7]
\end{tikzcd}\]
We take the morphism $y \rightarrow \text{cone}(\beta )$ in the above diagram to be $g$.  
\end{proof}
\end{document}
